\chapter{GİRİŞ}
Yıldız Teknik Üniversitesi öğrencilerine yönelik iş ve staj ilanları, kampüslerin çeşitli
panolarında yayımlanmaktadır. Bu ilanlar sadece bir bölümün öğrencilerine hitap
edebildiği gibi, tüm öğrencileri ilgilendiren ilanlar da olabilmektedir. \\
E-posta yoluyla iletilen ilanlar önce idare tarafından kontrol edilir. Eğer onaylanırsa
panolarda veya internet sayfaları üzerinden yayımlanır. İlanların yayımlanmadan önce
idare tarafından onaylanması zorunlu olduğundan dolayı, panolarda yer alan ilanların
kontrol edilmesi, onaylanmış olmayanların toplanması gerekmektedir. İnternet sayfaları
üzerinden yayımlanan ilanların kontrol edilmesi kolay olsa da panolarda yayımlanan
ilanların uygunluklarının düzenli kontrol edilmesine olanak yoktur. Ayrıca her iki
şekilde de ilanı yayımlayanlar, ilanın geri çekilmesi için bir teşebbüste bulunmamakta
bu yüzden geçerliliği kalmayan ilanlar halen göz önünde kalmakta, (özellikle son
geçerlilik tarihi bilgisi olmayan ilanlar) yanıltıcı olabilmektedir. \\
İlanların yönetim ve bakımı için bir zaman ayrılması, bu konuda bir emek harcanması
gerekmektedir. İlanların içeriğine göre gruplanması, internet sayfasına eklenmesi, son
başvuru tarihlerinin kontrol edilip geçerliliği kalmayan ilanların kaldırılması
gerekmektedir. İlanların yönetim ve bakımına ilişkin bu çabayı üstlenecek 
personelin kayda değer bir emek harcaması gerekmektedir. \\
Dağınık ilan yayınlama şeklinin önüne geçmek, kontrol edilebilir ilanlar yayımlatmak,
öğrencilere daha sağlıklı bir şekilde ilan sunmak bir ihtiyaç haline gelmiştir. Proje
kapsamında, bu ilan karmaşıklığını çözmek; yönetilebilir, denetlenebilir, çeşitli
parametrelerle filtrelenebilir ilanlar yayımlatmakla beraber; öğrencilerin sistem
üzerinden ilanlara başvurmalarına olanak sağlayacak bir web uygulaması geliştirilmesi
hedeflenmektedir. Bu web uygulaması Yıldız Teknik Üniversitesi öğrencileri
hedeflenerek yayımlanacak ilanların merkezi bir yönetimini sağlayacaktır. Diğer yandan
yayımlanacak ilanlar için bir bağış alınacak ve bu bağışlar öğrenciler ve öğrencilere yönelik hizmetler için harcanarak
eğitimlerine katkıda bulunulmuş olacaktır. Aynı zamanda Bilgisayar Mühendisliği
Bölümü gibi kendisine gelen ilanları kendi internet sayfasından yayımlayan bölümler
için, ilgili personelin iş yükü azaltılmış olacaktır.
