\chapter{ÖN İNCELEME}
İlan yayınlarının yapıldığı bazı web sayfaları/uygulamaları mevcuttur. Birçoğu işlevsel
olarak yeterliliğe sahip olsa da hedef kitlesi öğrenciler olarak tasarlanmadığından ele
alınan soruna çözüm olamamaktadır.

\section{Mevcut Çözümlerin İncelemesi}
Tablo 2.1' de ilan yayınlanan web uygulamalarından 6'sı karşılaştırmalı olarak
incelenmiştir. \\
Bu uygulamaların hepsi internet sayfaları aracılığıyla kullanıcılara ilan sunabilmektedir.
Bu sayfalardaki ilanların tek ortak özellikleri ise ilan başlığı ve ilan içeriğine yer vermiş
olmalarıdır. 1 numaralı uygulama, sadece YTÜ personelini ilgilendiren ilanları
yayımlamaktadır. 2 numaralı uygulama öğrencilere yönelik ilanlar barındırır. 1 ve 2
numaralı uygulamaların teknik yapıları aynıdır: Bir web sayfası üzerinden, yetki
gerektirmeksizin görüntülenebilen ilanlar. 3 ve 4 numaralı ilanlar ise YTÜ Bilgisayar
Mühendisliği öğrencileri hedeflenerek yayımlanan ilanlardır. Şirketler, özellikle YTÜ
Bilgisayar Mühendisliği öğrencileri için bu ilanların yayımlanmasını istese de, 1 ve 2
numaralı uygulamalarda olduğu gibi, 3 ve 4 numaralı uygulamalardaki ilanlar da
herhangi bir yetki gerektirmeksizin herkes tarafından görüntülenebilir. Bu şirketler
tarafından istenmeyen bir durumdur. 5 numaralı uygulama 1, 2, 3 ve 4 numaralı
uygulamalardan farklı olarak kullanıcı kaydı gerektirir ve ilanlar sadece kayıtlı
kullanıcılar tarafından görüntülenebilir. 6 numaralı uygulama bu açıdan 5 numaralı
uygulamaya benzerdir. Farklılık ise hedef kitlelerindedir. 5 numaralı uygulama sadece
YTÜ mezun öğrencilerine yönelik iken, 6 numaralı uygulamaya herkes kayıt olabilir. \\
Web uygulamaları her kullanıcıya aynı arayüzü sunan statik bir yapıda olabileceği gibi,
kullanıcılardan aldığı bilgilerle, veritabanından sorgulama yaparak oluşturduğu
sonuçları sunarak çalışan dinamik bir yapıda da olabilir. Dinamik yapıda olan web
uygulamaları çoğunlukla kullanıcı kaydı almakta ve kullanıcıların sayfalarını
özelleştirmelerine izin vermektedir. Tablo 2.1’deki örnek 5 ve 6 numaralı uygulamalar
dinamik web uygulamaları olup, gerçekleştirilmesi hedeflenen İlan Yayınlama Servisi
projesinde de bu şekilde tasarlanacaktır. Ayrıca proje, öğrenci odaklı, şirket ilanlarına
açık, üniversite özelinde olacaktır. Bu açıdan Tablo 2.1’deki uygulamaların ortak bir
paydada birleşmiş hali olacağı söylenebilir.

% Please add the following required packages to your document preamble:
% \usepackage{graphicx}
\begin{table}[]
\caption{Bazı ilan yayınlanan web Sayfalarının, problemin çözümüne yönelik yetkinlik durumları}
\label{my-label}
\resizebox{\textwidth}{!}{%
\begin{tabular}{ccccccc}
\hline
\multicolumn{1}{|c|}{NO} & \multicolumn{1}{c|}{\begin{tabular}[c]{@{}c@{}}İlanın Yapıldığı\\ Yer\end{tabular}}                                   & \multicolumn{1}{c|}{\begin{tabular}[c]{@{}c@{}}Öğrenciye \\ Özel? Mi\end{tabular}} & \multicolumn{1}{c|}{\begin{tabular}[c]{@{}c@{}}Üniversite \\ Özeli Mi?\end{tabular}} & \multicolumn{1}{c|}{İlan Kapsamı}                                                                              & \multicolumn{1}{c|}{\begin{tabular}[c]{@{}c@{}}Firma\\ İlanları\end{tabular}} & \multicolumn{1}{c|}{\begin{tabular}[c]{@{}c@{}}Sistem\\ Üzerinden\\ Başvuru\end{tabular}} \\ \hline
\multicolumn{1}{|c|}{1}  & \multicolumn{1}{c|}{\begin{tabular}[c]{@{}c@{}}YTÜ personel \\ sayfası\cite{ytuPersonel}\end{tabular}}                                  & \multicolumn{1}{c|}{Hayır}                                                         & \multicolumn{1}{c|}{Evet}                                                            & \multicolumn{1}{c|}{\begin{tabular}[c]{@{}c@{}}YTÜ personeli\\ bilgilendirme \\ amaçlı\end{tabular}}           & \multicolumn{1}{c|}{Hayır}                                                    & \multicolumn{1}{c|}{Hayır}                                                                \\ \hline
\multicolumn{1}{|c|}{2}  & \multicolumn{1}{c|}{\begin{tabular}[c]{@{}c@{}}YTÜ resmi \\ web sayfası\cite{ytuResmi}\end{tabular}}                                 & \multicolumn{1}{c|}{Hayır}                                                         & \multicolumn{1}{c|}{Evet}                                                            & \multicolumn{1}{c|}{\begin{tabular}[c]{@{}c@{}}Üniversite kapsamında\\ yer alan ilanlar\end{tabular}}          & \multicolumn{1}{c|}{Hayır}                                                    & \multicolumn{1}{c|}{Hayır}                                                                \\ \hline
\multicolumn{1}{|c|}{3}  & \multicolumn{1}{c|}{\begin{tabular}[c]{@{}c@{}}YTÜ BM\\ staj ilan sayfası\cite{bmStaj}\end{tabular}}                               & \multicolumn{1}{c|}{Evet}                                                          & \multicolumn{1}{c|}{Hayır}                                                           & \multicolumn{1}{c|}{\begin{tabular}[c]{@{}c@{}}BM öğrencilerine \\ yönelik staj ilanları\end{tabular}}         & \multicolumn{1}{c|}{Evet}                                                     & \multicolumn{1}{c|}{Hayır}                                                                \\ \hline
\multicolumn{1}{|c|}{4}  & \multicolumn{1}{c|}{\begin{tabular}[c]{@{}c@{}}YTÜ BM \\ İş İlan Sayfası\cite{bmIs}\end{tabular}}                                & \multicolumn{1}{c|}{Evet}                                                          & \multicolumn{1}{c|}{Hayır}                                                           & \multicolumn{1}{c|}{\begin{tabular}[c]{@{}c@{}}BM öğrencilerine \\ yönelik iş ilanları\end{tabular}}           & \multicolumn{1}{c|}{Evet}                                                     & \multicolumn{1}{c|}{Evet}                                                                 \\ \hline
\multicolumn{1}{|c|}{5}  & \multicolumn{1}{c|}{\begin{tabular}[c]{@{}c@{}}YTÜ Mezun \\ internet sitesi\cite{ytuMezun}\end{tabular}}                             & \multicolumn{1}{c|}{Hayır}                                                         & \multicolumn{1}{c|}{\begin{tabular}[c]{@{}c@{}}Sadece \\ mezunlar\end{tabular}}      & \multicolumn{1}{c|}{\begin{tabular}[c]{@{}c@{}}YTÜ mezunlarına \\ yönelik iş ve staj \\ ilanları\end{tabular}} & \multicolumn{1}{c|}{Evet}                                                     & \multicolumn{1}{c|}{Evet}                                                                 \\ \hline
\multicolumn{1}{|c|}{6}  & \multicolumn{1}{c|}{\begin{tabular}[c]{@{}c@{}}Özel bir şirkete \\ ait iş/staj ilanı \\ internet sitesi\cite{kariyerNet}\end{tabular}} & \multicolumn{1}{c|}{Hayır}                                                         & \multicolumn{1}{c|}{Hayır}                                                           & \multicolumn{1}{c|}{\begin{tabular}[c]{@{}c@{}}Herkese açık\\  iş ve staj ilanları\end{tabular}}               & \multicolumn{1}{c|}{Evet}                                                     & \multicolumn{1}{c|}{Evet}                                                                 \\ \hline
\multicolumn{1}{l}{}     & \multicolumn{1}{l}{}                                                                                                  & \multicolumn{1}{l}{}                                                               & \multicolumn{1}{l}{}                                                                 & \multicolumn{1}{l}{}                                                                                           & \multicolumn{1}{l}{}                                                          & \multicolumn{1}{l}{}                                                                      \\
\multicolumn{1}{l}{}     & \multicolumn{1}{l}{}                                                                                                  & \multicolumn{1}{l}{}                                                               & \multicolumn{1}{l}{}                                                                 & \multicolumn{1}{l}{}                                                                                           & \multicolumn{1}{l}{}                                                          & \multicolumn{1}{l}{}                                                                      \\
\multicolumn{1}{l}{}     & \multicolumn{1}{l}{}                                                                                                  & \multicolumn{1}{l}{}                                                               & \multicolumn{1}{l}{}                                                                 & \multicolumn{1}{l}{}                                                                                           & \multicolumn{1}{l}{}                                                          & \multicolumn{1}{l}{}                                                                      \\
\multicolumn{1}{l}{}     & \multicolumn{1}{l}{}                                                                                                  & \multicolumn{1}{l}{}                                                               & \multicolumn{1}{l}{}                                                                 & \multicolumn{1}{l}{}                                                                                           & \multicolumn{1}{l}{}                                                          & \multicolumn{1}{l}{}                                                                     
\end{tabular}%
}
\end{table}



\section{Tasarlanan Çözümün Genel Yapısı}

İlan Yayınlama Servisi projesinde yer alacak özellikler genel hatlarıyla şöyledir:
\begin{itemize}
    \item Şirket ve öğrencilerin sisteme kayıt olmaları
    \item Şirketlerin ilanları parametrelere göre hazırlaması, buna dayalı olarak
    \item Şirketlerin ilanları parametrelere göre hazırlaması, buna dayalı olarak
öğrencilerin istedikleri niteliklerdeki ilanları arayabilmeleri ve istedikleri
ilanlara sistem üzerinden başvurabilmeleri
    \item Uygunsuz içeriğe sahip ilanların şikayet edilmesine yönelik bir ceza sisteminin
oluşturulması
    \item Şirketlerin ilan yayınlamak için YTÜ’ nün vakıflarına bağış yapması ve bu şekilde elde edilen gelirin
harcanmasına yönelik kayıt tutulması
\end{itemize}

Bu özelliklerden hareketle sistem için 4 tip kullanıcı belirlenmiştir:
\begin{itemize}
    \item Sistem Yöneticisi
    \item Bağış Kabul Edebilen Vakıflar
    \item İlan yayımlamak isteyenler 
    \item İlanları inceleyecek olan öğrenciler
\end{itemize}

Bu kullanıcılardan sadece sistem yöneticisi 1 tane olacaktır, diğer kullanıcı tiplerinden
birden fazla olabilecektir. Şirket ve öğrenci kullanıcıları sisteme kayıt olacak, bağış
kabul birimi kullanıcıları ise sistem yöneticisi tarafından eklenecektir. \\
Hedeflenen bu çözüm ile öğrenci odaklı bir İlan Yayınlama Servisi oluşturulacaktır.
Şirket bağışlarından gelir elde edilecek, elde edilen bu gelir öğrenci odaklı çeşitli
harcamalarda kullanılabilecek ve bu harcamaların kaydı sistem üzerinden takip
edilebilecektir. En önemlisi ise ilanlar merkezi bir şekilde yönetilebilecektir.
